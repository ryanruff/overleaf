\documentclass[margin,line]{res}


\oddsidemargin -.5in
\evensidemargin -.5in
\textwidth=6.0in
\itemsep=0in
\parsep=0in
% if using pdflatex:
%\setlength{\pdfpagewidth}{\paperwidth}
%\setlength{\pdfpageheight}{\paperheight} 

\newenvironment{list1}{
  \begin{list}{\ding{113}}{%
      \setlength{\itemsep}{0in}
      \setlength{\parsep}{0in} \setlength{\parskip}{0in}
      \setlength{\topsep}{0in} \setlength{\partopsep}{0in} 
      \setlength{\leftmargin}{0.17in}}}{\end{list}}
\newenvironment{list2}{
  \begin{list}{$\bullet$}{%
      \setlength{\itemsep}{0in}
      \setlength{\parsep}{0in} \setlength{\parskip}{0in}
      \setlength{\topsep}{0in} \setlength{\partopsep}{0in} 
      \setlength{\leftmargin}{0.2in}}}{\end{list}}

%\usepackage[T2]{fontenc}
%\usepackage[utf8]{inputenc}

%\usepackage{tgbonum}

\begin{document}
\name{Ryan Richard Ruff, MPH, PhD \vspace*{.1in}} 

\begin{resume}
\section{\sc Contact Information}
\vspace{.05in}
\begin{tabular}{@{}p{3.4in}p{3in}}
433 First Avenue, Room 712 & {\it Voice:}  (212) 998-9663 \\            
Department of Epidemiology \& Health Promotion  & {\it E-mail:}  ryan.ruff@nyu.edu\\       
New York University College of Dentistry & {\it Web:} www.ryanruff.com \\ 
\end{tabular}


\section{\sc Research Interests}
Oral health and child development, specifically academic achievement and psychosocial functioning, intermittent fasting and oral health, statistical epidemiology, education policy

\section{\sc Education}
{\bf Harvard University}, Cambridge, MA USA\\
%{\em Department of Mathematics and Statistics} 
\vspace*{-.1in}
\begin{list1}
\item[] MPH in Epidemiology,  May 2015
\end{list1}

{\bf University of Virginia}, Charlottesville, VA USA\\
%{\em Department of Statistics} 
\vspace*{-.1in}
\begin{list1}
\item[] PhD in Research, Statistics, \& Evaluation, December 2010 
%\begin{list2}
%\vspace*{.05in}
%\item Dissertation Topic:  ``Nonstationary Covariance Models for
%  Spatial Data and Regression Problems'' 
%\item Dissertation Topic:  ``Hierarchical Models for Multiple Ratings
%  in Performance-Based\\ \hspace*{1.23in} Student Assessments.'' 
%\item Advisor:  Mark J. Schervish
%\end{list2}
\vspace*{.05in}
\item[] BAMT in Education,  May 2005
\end{list1}

{\bf Cambridge University}, Cambridge, UK\\
%{\em Department of Mathematics and Statistics} 
\vspace*{-.1in}
\begin{list1}
\item[] MPhil in Education, July 2010
\end{list1}


\section{\sc Honors and Awards} 
2020 Goddard Junior Faculty Fellow

\vspace*{-2.5mm}
2020 Dean’s Honor, NYU College of Dentistry (2020, 2018, and 2017)

\vspace*{-2.5mm}
2019 Paper selected as one of five “Highlights of 2018” in BMC Oral Health 

\vspace*{-2.5mm}
2018 Delta Omega Honor Society, New York University

\vspace*{-2.5mm}
2018 NYU Faculty Honoree, New York University

\vspace*{-2.5mm}
2017 Elected to Academy of Distinguished Educators, NYU College of Dentistry 

\vspace*{-2.5mm}
2012 Distinguished Service Award, NYC Department of Health

\vspace*{-2.5mm}
2008 DuPont Fellow, University of Virginia

\vspace*{-2.5mm}
2007 Doctoral Fellowship, University of Virginia (2007 and 2006)

%\vspace*{-2.5mm}
%NSF Vertical Integration of Research and Education in Statistics and
%Mathematical Sciences\\ (VIGRE) teaching fellowship.
%

\section{\sc Academic Experience}
{\bf New York University College of Dentistry}, New York, NY

\vspace{-.3cm}
{\em Associate Professor (tenure track)} \hfill {\bf 2019 - Pres} \\
\vspace*{-.2cm}
\begin{list2}
\item Co-Director, Oral Health Equity and Action Lab, 2019 - Pres
\item Director, Biostatistics Core, 2017 - Pres
\item Director, MS Program in Clinical Research, 2017 - Pres
\end{list2}
%Includes current Ph.D.~research, Ph.D.~and Masters level coursework and
%research/consulting projects.

\vspace{-.2cm}
{\em Assistant Professor} \hfill {\bf 2017 - 2019}

\vspace{-.2cm}
{\em Research Assistant Professor} \hfill {\bf 2014 - 2017}\\
\vspace{-.3cm}

{\bf New York University School of Global Public Health}, New York, NY

\vspace{-.3cm}
{Associated Professor} \hfill {\bf 2014 - Pres}

\vspace{-.3cm}
{Adjunct Associated Professor} \hfill {\bf 2012 - 2014}

%\vspace{-.1cm}
%{\em NSF VIGRE Teaching Fellow} \hfill {\bf January - May, 2001}\\
%Head teaching assistant.   
%Duties included  shared administrative responsibilities with faculty
%instructor, fielding of all student inquiries, and oversight of
%graduate student teaching assistants and graders.
%\vspace*{.05in}  
%\begin{list2}
%\item 36-217 Probability Theory and Random Processes, Spring 2001.
%\end{list2}

\section{\sc Professional Experience}
{\bf Columbia University, Epidemiology \& Health Institute}, New York, NY

\vspace{-.3cm}
{\em Instructor} \hfill {\bf 2012 - 2020}\\
\vspace{-.3cm}

{\bf New York City Department of Health}, New York, NY

\vspace{-.3cm}
{\em Director, Research \& Evaluation Unit} \hfill {\bf 2011 - 2014}\\
\vspace{-.3cm}

{\bf New York City Department of Education}, New York, NY

\vspace{-.3cm}
{\em Director of Research \& Evaluation}, New York, NY \hfill {\bf 2010 - 2011}\\
\vspace{-.3cm}


\section{\sc Research Experience}
{\bf Funded Research - As Principal Investigator} \hfill 

\vspace*{-2.5mm}
New York University Mega Grant Support Initiative \#RA633, “The role of the oral microbiome in disparities in dental caries and responsiveness to caries prevention: A pilot study.” \$15,000, 2019 - 2020

Patient Centered Outcomes Research Institute \#PCS160936724, “Comparative effectiveness of school-based caries prevention for children: A cluster randomized pragmatic clinical trial.” \$13,352,988, 2018 - 2023

National Institute on Minority Health and Health Disparities \#R01 MD011526, “Statewide implementation and evaluation of a rural caries prevention program.” \$3,617,644, 2017 - 2022

National Institute of Dental and Craniofacial Research \#R03 DE025289, “Analysis of longitudinal data from a school-based caries prevention program.” \$317,000, 2016 - 2018

{\bf As co-investigator or other role} \hfill 

\vspace*{-2.5mm}
National Institute of Dental and Craniofacial Research \#R01 DE018729, “Quality of life in children with cleft.” \$3,231,498 (Hillary Broder PI), 2015 - 2016	

National Institute on Minority Health and Health Disparities \#U24 MD006964, “Effectiveness and improvement of rural, school-based, caries prevention programs.” \$7,100,000 (Richard Niederman PI), 2014 - 2015	

NYU Community and Translational Science Institute, “Understanding obesity among public housing residents in NYC: A community-based geospatial project.” \$35,000 (Dustin Duncan PI), 2013 - 2014

Robert Wood Johnson Foundation \#71637, “Assessing the impact of food restrictions under the Supplemental Nutrition Assistance Program on food choice by families.” \$170,000 (Chen Zhen PI), 2013 - 2014

{\bf Grants under review} \hfill 

\vspace*{-2.5mm}
National Institute of Dental and Craniofacial Research, “The role of the oral microbiome in disparities in dental caries and responsiveness to caries prevention” (PI, \$3,384,191)

National Institute of Dental and Craniofacial Research, “EPI-MOUTH: An epidemiologic dental model for simulating oral health interventions.” (PI, \$304,629)
		
National Institute of Dental and Craniofacial Research, “Intermittent fasting, dental health, and quality of life: The IFAD Study.” (PI, \$431,000)

{\bf Other funded research} \hfill 

\vspace*{-2.5mm}
New Hampshire Women, Infants, and Children Pay for Prevention Program \\
Epidemiologist/Biostatistician (\$9,125).
\begin{list2}
\item Analyzed data from a longitudinal caries prevention program amongst pregnant women and children in rural NH counties, 2018
\end{list2}

New York City Department of Health and Mental Hygiene \\
Consultant (\$40,000).
\begin{list2}
\item Subject-matter expert regarding an oral health data registry and survey system, 2017-2018	
\end{list2}


\section{\sc Publications}
%\begin{enumerate}
%  \item The labels consists of sequential numbers.
%  \item The numbers starts at 1 with every call to the enumerate environment.
%\end{enumerate}

Khouly I, Pardinas-Lopez S, Ruff RR, Strauss FJ. Efficacy of growth factors for the
treatment of peri-implant diseases: A systematic review and meta-analysis. Clinical Oral Investigations. 2020. DOI: 10.1007/s00784-020-03240-5.

Ruff RR, Saxena D, Niederman R. School-based caries prevention and longitudinal trends in untreated decay: An updated analysis with Markov chains. BMC Research Notes. 2020;13(25). DOI: 10:1186/s13104-020-4886-8.

Ruff RR. Street-level research in public health nutrition: A cross-sectional study of dietary behavior in New York City. SAGE Research Methods Cases: Medicine \& Health. 2020. DOI: 10.4135/
9781529709698.

Qi L, Pushalkar S, Kamer A, Corby P, Mota R, Paul B, Guo Y, Ruff RR, Alekseyenko A, Li X, Saxena D. The influences of bioinformatics tools and reference databases in analyzing human saliva microbial community. Genes. 2020. In press.

Yang C, Crystal Y, Ruff RR, Niederman R. Quality appraisal of child oral health-related quality of life measures: A scoping review. JDR Clinical \& Translational Research. 2020;5(2). DOI: 10.1177/2380084419855636.

Huang S, Ruff RR, Niederman R. An economic evaluation of a comprehensive school-based caries prevention program. JDR Clinical \& Translational Research. 2019;4(4):378-387. DOI:  10.1177/
2380084419837587.

Ruff RR. State-level autonomy in the era of accountability: A comparative analysis of Virginia and Nebraska education policy through No Child Left Behind. Education Policy Analysis Archives. 2019;27(6):1-31. DOI: 10.14507/epaa.27.4013.

Ruff RR, Senthi S, Susser S, Tsutsui A. Oral health, academic performance, and school absenteeism in children and adolescents: A systematic review and meta-analysis. J Am Dent Assoc. 2019;150(2):111-121. DOI: 10.1016/j.adaj.2018.09.023.

Ruff RR, Niederman R. Silver diamine fluoride versus therapeutic sealants for the arrest and prevention of dental caries in low-income minority children: Study protocol for a cluster randomized controlled trial. Trials. 2018;19(1):523. DOI: 10.1186/s13063-018-2891-1.

Ruff RR. Total observed caries experience: Assessing the effectiveness of community-based caries prevention. Journal of Public Health Dentistry. 2018;78(4):287-290. DOI: 10.1111/ jphd.12284.

Ruff RR, Niederman R. School-based caries prevention, tooth decay, and the community environment. JDR Clinical \& Translational Research. 2018;3(2):180-187. DOI: 10.1177/ 2380084417750612.

Ruff RR, Niederman R. Comparative effectiveness of school-based caries prevention: A prospective cohort study. BMC Oral Health. 2018;18(1):53. DOI: 10.1186/s12903-018-0514-6. 
{\em (Recognized as one of five “Highlights of 2018” published in journal)}.

Ruff RR, Crerand CE, Sischo L, et al. Surgical care for school-aged youth with cleft: Results from a multicenter, prospective observational study. The Cleft Palate-Craniofacial Journal. 2018;55(8):1166-1174. DOI: 10.1177/1055665618765776.

Ruff RR, Niederman R. Comparative effectiveness of treatments to prevent dental caries given to rural children in school-based settings: Protocol for a cluster randomized controlled trial. BMJ Open. 2018;8(4). DOI: 10.1136/bmjopen-2018-022646.

Ruff RR, Sischo L, Chinn CH, Broder HL. Development and validation of the Child Oral Health Impact Profile - Preschool version. Community Dental Health. 2017;34(3):176-182. DOI: 10.1922/
CDH\_4110Ruff07.

Huang SS, Ruff RR*, Niederman R. The benefit of early preventive dental care for children. JAMA Pediatrics. 2017;171(9):918. DOI:10.1001/jamapediatrics.2017.2066.

Broder HL, Crerand CE, Ruff RR, Peshansky A, Sarwer DB, Sischo L. Challenges in conducting multicenter, multidisciplinary, longitudinal studies in children with chronic conditions. Community Dentistry and Oral Epidemiology. 2017;45(4):1-6. DOI: 10.1111/cdoe.12293.

Rajendra A, Veitz-Keenan A, Oliveira BH, Ruff RR, Wong MCM, Innes NPT, Radford J, Seifo N, Niederman R. Topical silver diamine fluoride for managing dental caries in children and adults (Protocol). Cochran Database of Systematic Reviews. 2017;7. DOI: 10.1002/14651858.CD012718.

Ruff RR, Sischo L, Broder HL. Minimally important difference of the Child Oral Health Impact Profile for children with orofacial anomalies. Health and Quality of Life Outcomes. 2016;14(140). DOI: 10.1186/s12955-016-0544-1.

Ruff RR, Broder H, Sischo L. Resiliency and socioemotional functioning in youth receiving surgery for orofacial anomalies. Community Dentistry and Oral Epidemiology. 2016;44(4):371-80. DOI: 10.1111/cdoe.12222.

Ruff RR. The impacts of retention, expenditures, and class size on primary school completion in Sub-Saharan Africa: A cross-national analysis. International Journal of Education Policy and Leadership. 2016;11(8).

Ruff RR, Ng J, Jean-Louis G, Elbel B, Chaix B, Duncan DT. Neighborhood stigma is associated with poor sleep health in a pilot study of low-income housing residents in New York City. Behavioral Medicine. 2016;44(1). DOI: 10.1080/08964289.2016.1203754.

Duncan DT, Ruff RR, Chaix C, Regan SD, et al. Perceived spatial stigma, body mass index, and blood pressure: A GPS study among low-income housing residents in New York City. Geospatial Health. 2016;11(2). DOI: 10.4081/gh.2016.399

Ruff RR. Sugar-sweetened beverage consumption is linked to global adult mortality through diabetes mellitus and adiposity-related diseases including cardiovascular disease and cancer. Evidence Based Medicine. 2015;20(6):223-4. DOI: 10.1136/ebmed-2015-110267.

Ruff RR, Zhen C. Estimating the effects of a calorie-based sugar-sweetened beverage tax on weight and obesity in New York City adults using dynamic loss models. Annals of Epidemiology. 2015:25(5):
350-357. DOI: 10.1016/j.annepidem.2014.12.008.

Ruff RR, Adjoian T, Akhund A. Small convenience stores and the local food environment: An analysis of resident shopping behavior using multilevel modeling. American Journal of Health Promotion. 2015;30(3):172-180. DOI: 10.4278/ajph.140326-QUAN-121.

Yi SS, Ruff RR, Jung M, Waddell EN. Racial/ethnic residential segregation, neighborhood poverty and urinary biomarkers of diet in New York City adults. Social Science and Medicine. 2014;122:122-129. DOI: 10.1016/j.socscimed.2014.10.030.

Duncan DT, Regan SD, Shelley D, Day K, Ruff RR, Al-Bayan M, Elbel B. Using Global Positioning System (GPS) methods to study the spatial contexts of obesity and hypertension risk among low-income housing residents in New York City: A feasibility study. Geospatial Health. 2014;9(1):57-70.

Miller-Archie SA, Jordan HT, Ruff RR, Chamany S, Cone JE, Brackbill RM, et al. Posttraumatic stress disorder and new-onset diabetes among adult survivors of the World Trade Center disaster. Preventive Medicine. 2014;66:34-38. DOI: 10.1016/j.ypmed.2014.05.016

Zhen C, Brissette I, Ruff RR. By ounce or by calorie: The differential effects of alternative sugar-sweetened beverage tax strategies. American Journal of Agricultural Economics. 2014;96(4):1070-1083. DOI: 10.1093/ajae/aau052. Published online June 2, 2014.

Day K, Loh R, Ruff RR, Rosenblum R, Fischer S, Lee K. Does bus rapid transit promote walking? An examination of New York City’s Select Bus Service. Journal of Physical Activity and Health. 2014;11(8):1512-1516. DOI: 10.1123/jpah.2012-0382.

Ruff RR, Akhund A, Adjoian T, Kansagra SM. Calorie intake, sugar-sweetened beverage consumption, and obesity among New York City Adults: Findings from a 2013 population study using dietary recalls. Journal of Community Health. 2014;39(6):1117-1123. DOI: 10.1007/s10900-014-9865-3.

Ruff RR, Rosenblum R, Fischer S, Meghani H, Adamic J, Lee KK. Associations between building design, point-of-decision stair prompts, and stair use in urban worksites. Preventive Medicine. 2013;60:60-64. DOI: 10.1016/j.ypmed.2013.12.006.

Ruff RR. School counselor and school psychologist perceptions of accountability policy: Lessons from Virginia. The Qualitative Report. 2011;16(5):1270-1290.



\section{\sc Papers in preparation}
Ruff RR, Monse B, Duijster D, Naliponguit E, Benzian H. School-based strategies to
prevent tooth decay in a high-risk child population in the Philippines: Results of a longitudinal cohort study. Submitted, Journal of Dental Research.

Aldosari MA, Bukhari OM, Ruff RR, Douglass CW, Niederman R, Starr JR. Comprehensive, school-based dentistry: Program details and students’ unmet dental needs. Submitted, Journal of School Health.

Ruff RR, Whittemore R, Barry-Godin T. Systematic review and meta-analysis of dental
caries and quality of life in children and adolescents.

Xu F, Saxena D, Crystal Y, Ruff RR. Differential bacterial abundance in saliva in children exhibiting non-responsiveness to silver diamine fluoride for caries arrest: A pilot study.

Ruff RR, Sierra MA, Xu F, Crystal Y, Saxena D. Predicting treatment nonresponse in children receiving silver diamine fluoride using artificial neural networks: A pilot study.

Ibrahem F, Giugliano T, Goldstein GR, Ruff RR, Choi M. Digital analysis of the dimensional change of irreversible hydrocolloid impression.

Alsaeid W, Zhivago P, Zhang Y, Ruff RR, Giugliano T, Choi M. Fracture load comparison between 3D printed and milled provisional restoration.

Ruff RR, Niederman R. Silver diamine fluoride is non-inferior to therapeutic sealants in the two-year arrest of dental caries: Results from the CariedAway cluster randomized pragmatic trial.

Ruff RR, Niederman R. Silver diamine fluoride versus therapeutic sealants for the prevention of dental caries: Results from the CariedAway cluster randomized pragmatic trial.

Ruff RR. Nurses are non-inferior to hygienists in the retention and effectiveness of silver diamine fluoride for the arrest of dental caries in pragmatic settings.

Ruff RR. Oral health, academic achievement, and the impact of school-based caries prevention: Results from the CariedAway trial.

Ruff RR. Oral health, student attendance, and the impact of school-based caries prevention.

Ruff RR, Niederman R. Quality of life and dental caries in low-income minority children: Results from the CariedAway study.

Ruff RR, Niederman R. Semiannual versus annual treatment of silver diamine fluoride for the prevention of dental caries

Ruff RR, Niederman R. Quality of life and SAFE dentistry: A joint trajectory model.

Ruff RR. Group trajectories in dental caries in children: Results from the CariedAway trial.

Ruff RR, Barry-Godin T, Whittemore R, Reitz J, Santiago-Galvin N, Gibbs H, Niederman R. Implementation of a large cluster-randomized pragmatic trial for the prevention of dental caries in school-based settings: Lessons from the CariedAway study.


\section{\sc Conference Presentations}
Assessing additional options for school-based caries prevention (“Improving
determinants of child development” Symposium). Presented at the IADR World Congress on Preventive Dentistry, New Delhi, India, October 2017

Predicting treatment nonresponse in children receiving silver diamine fluoride: A pilot study. Oral presentation, 9th International Conference on Methodological Issues in Oral Health Research, New York, USA (canceled due to COVID-19), June 2020

Measuring responsiveness to preventative treatments for dental caries: Towards a predictive model using machine learning (Ruff RR, Saxena D). Oral presentation given at the CED-IADR Oral Health Research Congress, Madrid, Spain, September 2019

Applications of virtual reality in training oral and maxillofacial surgery techniques (Ruff RR). NYU Academy of Distinguished Educators teaching and learning club, New York, USA, June 2018.

Research to optimize school-based caries prevention (Ruff RR). Oral presentation given at the 8th International Conference on Methodological Issues in Oral Health Research, Hong Kong, May 2018

\section{\sc Abstracts}
Should dental insurers implement comprehensive school-based caries prevention programs? (Huang S, Ruff RR, Niederman R). 41st Annual Meeting of the Society for Medical Decision Making, October 2018

Development and validation of the Child Oral Health Impact Profile – Preschool Version (Broder HL, Sischo L, Ruff RR, et al). 75th Meeting of the American Cleft Palate Craniofacial Association, April 2018

Analysis of school-based caries prevention programs (Ruff RR). New York University Faculty Research Symposium, New York, USA, April 2017

CariedAway: Outcomes of school-based caries prevention (Ruff RR, Niederman R). 95th General Session of the IADR, California, USA, March 2017

Patient-centered outcomes in periodontitis patients (Rego R, Russell S, Burukunte S, Ruff RR, et al). 95th General Session of the IADR, California, USA, March 2017

SDF for Caries Prevention Primary Teeth: A systematic review (Oliveira B, Rajendra A, Keenan A, Ruff RR, Niederman R). 95th General Session of the IADR, California, USA, March 2017

Spatial stigma and drug use among low-income housing residents in NYC (Ruff RR, Duncan DT). New York City Epidemiology Forum, New York, NY, February 2015

Estimating the effects of sugar-sweetened beverage taxes on weight and obesity in NYC (Ruff RR, Zhen C). American College of Epidemiology annual meeting, Maryland, USA, September 2014

Dietary sources of sugar and drink-specific calorie intake in New York City
Adults (Adjoian T, Ruff RR, Akhund A). New York City Epidemiology Forum, February 2014

%Paciorek, C.J. and R. Rosenfeld.  2000.  Minimum classification error
%training in exponential language models.  2000 Spring Transcription
%Workshop, College Park, Maryland.
%\vspace*{-.25in}  
%\begin{verbatim}http://www.nist.gov/speech/publications/tw00/html/abstract.htm#cp1-50\end{verbatim}

\section{\sc Service \& Committee Activities} 
{\bf New York University} \hfill 

\vspace*{-2.5mm}
Spring 2018	College faculty representative, NYU Faculty Urban Research Day

\vspace*{-2.5mm}
2014-2016	Teaching fellow, Center for the Promotion of Research Involving Innovative 
Statistical Methodology (PRIISM)

{\bf College} \hfill 

\vspace*{-2.5mm}
2019-Pres	Faculty Council Representative 

\vspace*{-2.5mm}
2016-Pres	Director, MS Program in Clinical Research

\vspace*{-2.5mm}
2016-Pres	Director, Biostatistics Core

\vspace*{-2.5mm}
2017-Pres	Member, NYU Academy of Distinguished Educators

\vspace*{-2.5mm}
2018-2022	Member, Curriculum Committee 

\vspace*{-2.5mm}
2016-Pres	Member, Graduate Council on Ethics and Professionalism

\vspace*{-2.5mm}
2016-2019	Member, Council on Humanitarianism and Culture Change

\vspace*{-2.5mm}
2017-2018	Mentor, Summer D1 Research Experience

\vspace*{-2.5mm}
2014-2015 	Teaching and Learning Club

\vspace*{-2.5mm}
2014-2015	Statistics Unit Head, College of Dentistry and School of Nursing

\vspace*{-2.5mm}
2014-2016	Technology Enhanced Learning Committee (NYU SGPH)


{\bf Department} \hfill 

\vspace*{-2.5mm}
2019 Planning Committee, 2020 Methodology in Oral Health Research Conference

\vspace*{-2.5mm}
Spring 2017	Hiring Committee

\vspace*{-2.5mm}
2015-2016	Chair, Curriculum Committee, MS Program in Clinical Research 

\vspace*{-2.5mm}
2015-2016	Admissions Committee

\vspace*{-2.5mm}
2014-2015	Curriculum Committee 


{\bf Other} \hfill 

\vspace*{-2.5mm}
2011-2014	Surveillance and Epidemiology Unit, NYC Incident Command System

\vspace*{-2.5mm}
2010-2011	Member, NYC Department of Education Institutional Review Board


{\bf Student Advising} \hfill 

\vspace*{-2.5mm}
Analyzing the impact of prevention on new dental caries over time in schoolchildren with preexisting untreated decay: The Forsyth Study (Kajal Desai, MPH thesis - 2019)

%\vspace*{-2.5mm}
The comparative performance and appropriateness of caries indices for use in oral epidemiology research: a prospective cohort example (Dhawani Shah, MPH thesis - 2019)

%\vspace*{-2.5mm}
Factors related to periodontitis prevalence and dental care utilization in 
US Hispanic women with changes in progesterone and estrogen levels (Ifrah Choudhary, MPH thesis - 2017)

%\vspace*{-2.5mm}
Periodontitis and the risk of psoriasis in the US population using the National Health and Nutrition Examination Survey, 2009-2012 (Zhen Li, MS capstone - 2016)


{\bf Peer and Editorial Review} \hfill 

\vspace*{-2.5mm}
{\em Reviewer}: PLOS One, BMJ Open, Preventive Medicine, JDR Clinical and Translational Research, BMC Oral Health, Journal of Public Health Dentistry, Community Dentistry \& Oral Epidemiology, Evidence-Based Medicine, BMC Public Health, Journal of Clinical Periodontology, Journal of the American Dental Association, Medicine, International Dental Journal, Clinical Implant Dentistry and Related Research

\vspace*{-2.5mm}
{\em Editor}: Medicine



{\bf Community Service} \hfill 

\vspace*{-2.5mm}
2017 Grant Reviewer, Oral Disease and Dental Health Review Committee, California Tobacco-Related Research Program (TRDRP). Ccontent expert for oral disease epidemiology and biostatistics.

2014 Consulting, Fiji National University, for the 2014 Fijian Oral Health Survey. Sampling procedures, data management and analysis, and reporting for government release.


\section{\sc Educational Activities} 
{\bf Courses Taught} \hfill 

\vspace*{-2.5mm}

\begin{tabular}{@{}p{4.5in}p{2in}}
Scientific Writing (MS), 15 hours            & Spring 2020 \\            
Epidemiology of Dental Caries (DDS, "Diagnosis and Treatment of Oral Disease" course) & 2018 - 2020 \\	
Biostatistics for Public Health (MPH/PhD/DDS), 45 hours & 2014 - 2019 \\
Data Management using SAS I (MS), 30 hours & 2014 - 2015 \\
Data Management using SAS II (MS), 30 hours & 2014 - 2015 \\
Clinical Research Practicum I (MS), 45 hours per semester & 2016 - 2020 \\
Clinical Research Practicum II (MS), 45 hours per semester & 2016 - 2020 \\
Biostatistics I (MPH/PhD), 45 hours per semester & 2012 - 2014 \\	
Biostatistics II (MPH/PhD), 45 hours per semester & 2012 - 2014	\\
Propensity Score Matching (Post-graduate), 15 hours & 2014 - 2020 \\	
Methods of Causal Inference (Post-graduate), 35 hours & Summer 2013 \\
Survey Research (Post-graduate, co-taught), 5 hours & Summer 2012 \\
Qualitative Data Analysis (PhD) 45 hours per semester & Spring 2009 \\
\end{tabular}

{\bf Educational Materials Developed} \hfill 

\vspace*{-2.5mm}

{\em MS Program in Clinical Research}, 2016 \\
In 2016 I was asked to assume the role of Director of the MS Program in Clinical Research in the department of Epidemiology and Health Promotion. In so doing, I evaluated and revised the program curriculum, moving from 50 credits to 30, updated all syllabi for new comptencies and state requirements, and successfully applied for state approval. I am responsible for reviewing and admitting outstanding applicants and developing new educational materials each year including advertising and new program options, such as a dual MS/MS with Biomaterials and a joint BS/MS with the NYU College of Arts and Sciences.

{\em Biostatistics for Public Health}, 2014 \\
In collaboration with the NYU Global Technology Service and the College of Global Public Health, I created a fully-online course in Biostatistics to be offered to students at CGPH, the College of Dentistry, and the School of Medicine. This project required creating a new syllabus, all educational materials (including lessons, datasets, problem sets, and lab sheets), writing lecture scripts, and recording professional-quality videos for use in the online classroom. 

{\em Data Management II}, 2014 \\
For Data Management II, I created a new syllabus with updated competencies and focused the core course on a set of skills necessary for students pursuing careers in Clinical Research. Development for this course included a new syllabus, writing new SAS code for every course module, and creating problem sets/projects for student assessment. 

{\em Propensity Score Matching}, 2012	\\
The PSM course is a fully online course using the Canvas Instructure framework. It is a five-module course on the theory and practice of causal inference and propensity scores to analyze epidemiologic data. Development included syllabus, lessons, lecture videos, datasets, laboratory problem sets, and worksheets. 


\section{\sc Membership in professional organizations} 
\begin{list2}
\item Member, Cochrane Collaboration
\item Fellow, Royal Society of Medicine
\item Member, New York Academy of Sciences
\item Fellow, Royal Society of Public Health
\item Member, International Association of Dental Research
\item Member, American Dental Education Association
\end{list2}


\section{\sc Software} 
\begin{list2}
\item Statistical Packages:  R, Stata, LISREL, AMOS, BUGS; some experience
  with SAS; SPSS.
\item Applications: EpiMouth (built in R Shiny), \LaTeX, N'Vivo, EndNote, JMP, common Windows
  database, spreadsheet, and presentation software
\end{list2}


\section{\sc Brief Research Statement}
My research in oral health epidemiology focuses on (1) oral health and child development 
and (2) intermittent fasting, oral health, and quality of life. In oral health and child 
development, I primarily study the association between school-based caries prevention and 
student academic achievement. This has included serving as principal investigator for 
several mega-grant initiatives conducting pragmatic cluster randomized trials on the non-
inferiority of silver diamine fluoride for the arrest and prevention of dental caries. It also 
includes partnering with microbiologists to study the multilevel determinants of non-
responsiveness to caries prevention. Additional areas of research in this field include the 
psychosocial development of children with cleft lip and palate and the measurement of oral 
health-related quality of life in children. 

While intermittent fasting has been studied extensively in relation to obesity and cardiometabolic outcomes, research on fasting and oral health or quality of life is virtually nonexistent. In this line of inquiry, my research focuses on the changes in oral health-related quality of life in response to calorie restricted feeding, the changes experienced in the oral microbiome, and the incidence of caries and periodontitis in long-term fasters. This research connects my prior experience as a research scientist at the NYC Department of Health, where I focused on chronic disease prevention and obesity research.




\end{resume}
\end{document}




